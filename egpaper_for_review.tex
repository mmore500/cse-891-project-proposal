%\title{Cdiscount Project Proposal}
\documentclass[10pt,twocolumn,letterpaper]{article}

\usepackage{cvpr}
\usepackage{times}
\usepackage{epsfig}
\usepackage{graphicx}
\usepackage{amsmath}
\usepackage{amssymb}

% Include other packages here, before hyperref.

% If you comment hyperref and then uncomment it, you should delete
% egpaper.aux before re-running latex.  (Or just hit 'q' on the first latex
% run, let it finish, and you should be clear).
\usepackage[pagebackref=true,breaklinks=true,letterpaper=true,colorlinks,bookmarks=false]{hyperref}

\cvprfinalcopy % *** Uncomment this line for the final submission

\def\cvprPaperID{****} % *** Enter the CVPR Paper ID here
\def\httilde{\mbox{\tt\raisebox{-.5ex}{\symbol{126}}}}

% Pages are numbered in submission mode, and unnumbered in camera-ready
\ifcvprfinal\pagestyle{empty}\fi
\begin{document}

%%%%%%%%% TITLE
\title{Cdiscount Image Categorization}

\author{Steven Jorgensen\\
Michigan State University \\
{\tt\small jorgen72@mu.edu}
% For a paper whose authors are all at the same institution,
% omit the following lines up until the closing ``}''.
% Additional authors and addresses can be added with ``\and'',
% just like the second author.
% To save space, use either the email address or home page, not both
\and
Matthew Andres Moreno\\
Michigan State University\\
{\tt\small mmore500@msu.edu}
\and
Ian Whalen\\
Michigan State University \\
{\tt\small whalenia@msu.edu}
}


\maketitle
%\thispagestyle{empty}

%%%%%%%%% ABSTRACT
\begin{abstract}
Online European retailer Cdiscount maintains an online catalog of 30 million products.
Cdiscount could achieve greater efficiency in managing its catalog of products by automating the product classificiation process. 
Cdiscount organizes its products into more than 5,000 categories, posing an extreme multi-category classification scheme problem.
We aim to develop an approach that leverages data made available by the retailer to allow for accurate, automated cataloging of products based on their images.
We will use existing convolutional neural network techniques (CNN) to classify images while exploiting the hierarchical nature of the Cdiscount categorization scheme to reduce the complexity of the problem.
\end{abstract}

%%%%%%%%% BODY TEXT
\section{Problem Description}

Cdiscount is one of France's largest e-commerce companies, selling large items such as trampolines, to small items like Tvs.
As the company grows, so to does the number of different products they offer.
In the past 2 years alone, they have added over 20 million new products to their wares.\cite{cDiscountKaggle} 
Ensuring that each and every product is classified correctly is a challenging task.
Currently, Cdiscount uses machine learning text classification methods to categorize their wares.
To improve the accuracy of their classification methods, they would like to begin using images, rather than text, to categorize products.
The data set provided has over 15 million images ranging over 5,000 categories, and each image has 3 different levels of categorization, making this a challenging multi-classification problem.

\section{Proposed Approach}

Our approach will be based on standard convolutional neural network techniques used perform multiclass image categorization \cite{Krizhevsky2012}.
Exploiting the hierarchical nature of Cdiscount's classification scheme will be key to achieving good performance despite the large number of categories \cite{deng2010does}.
We will try several approaches to exploit the hierarchical nature of the category scheme employed by Cdiscount.
We will test nested image classification models, where separate networks are trained to sort images between categories at each hierarchical level.
We will also test schemes that less severely penalize misclassification of an image to a category closely related to it in the product categorization hierarchy and schemes that are trained to perform categorization on all three levels of hierarchy simultaneously.

We will use the TensorFlow framework to implement our models.

\section{Data}

Cdiscount has made an extensive dataset available through the data science platform Kaggle.
This dataset includes the full hierarchical categorization scheme used by Cdiscount and a listing of over 9 million products.
Each product has a unique ID, the ID of the category it falls in, and one to four images of that product.
The dataset comes divided into training and testing datasets, with the training set describing approximately 7 million products and the testing set describing approximately 2 million products.

\section{Division of Labor}

Ian will lead laying out and implementing the network architecture.
Steven will focus on organizing our data for use in training and assessing our model's performance with validation data.
Matthew will design and implement the training technique and coordinate training using high performance computing assets.

We will meet at least weekly to discuss and assess progress and work on the project together.


\section{Milestones}


\textit{Midterm report --- November 12, 2017}

By the midterm report we will have a fully implemented solution for high level classifications and will assess the performance of our model on high level classification.
By this time, we will also have a set of candidate strategies for performing lower level classifications to evaluate.\\


\textit{Final report --- December 13, 2017}

When we reach the final report, we hope to have a fully implemented model to organize products at all three levels of categorization.
We will use cross validation assess and report the performance of our model on image classification at all hierarchical levels.


{\small
\bibliographystyle{ieee}
\bibliography{egbib}
}

\end{document}
